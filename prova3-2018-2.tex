\documentclass[12pt]{article}

\usepackage[utf8]{inputenc}
\usepackage[brazil]{babel}
\usepackage{amsmath,amssymb}
\usepackage{pdfsync}
\usepackage[all]{xy}
\usepackage{color}

\newcommand{\resposta}[1]{ \noindent {\bf Solução}.{\color{blue} #1}}

\title{{\large Universidade de Brasília \\ Instituto de Ciências Exatas \\
Departamento de Ciência da Computação} \\[1cm]
CIC 117536 - Projeto e Análise de Algoritmos \\[.5cm]  Terceira Prova \\[.5cm] Turma: B}
\author{{\bf NP-completude}}
\date{Prof. Flávio L. C. de Moura \\[.5cm] \today}

\begin{document}
\maketitle

\begin{enumerate}
\item {\bf (2.5 pontos)} O problema 2-SAT tem como instâncias as
  fórmulas lógicas formadas por conjunções de disjunções de até dois
  literais, onde um literal é uma variável booleana ou a negação de
  uma variável booleana. Por exemplo, a expressão a seguir é uma
  instância de 2-SAT:

  $$(x_1\lor \neg x_2)\land (\neg x_1 \lor \neg x_3) \land (x_1 \lor x_2) \land x_3$$

  Prove que 2-SAT $\in$ P.

 
  \resposta{
    Consideremos a seguinte expressão cuja é uma instância de 2-SAT:
    
    $$(\neg x_1 \lor x_2) \land (\neg x_2 \lor x_3) \land (x_1 \lor \neg x_3) \land (x_3 \lor x_2)$$
    
    Dessa forma, obtemos duas afirmações:
    
    \begin{enumerate}
        \item Se G contém um caminho de $\phi$ até $\psi$, então também contém o caminho de $\neg \psi$ até $\neg \phi$.
        
        Para provarmos, consideremos que o caminho de $\phi$ a $\psi$ é $\phi \rightarrow P_1 \rightarrow ... \rightarrow P_k \rightarrow \psi$. Agora, pela construção de G, se há uma aresta $(x, y)$, então tem um aresta $(\neg x, \neg y)$. Logo, temos $(\neg \psi, \neg P_k), ..., (\neg P_1, \neg \phi)$. Portanto, temos um caminho de $\neg \psi$ até $\neg \phi$.
        
        \item Uma cláusula conjuntiva de dois valores $\Upsilon$ é insatisfazível se e somente se existe um $x$ tal que:
        
        \begin{enumerate}
            \item há um caminho de $x$ a $\neg x$ no grafo.
            \item há um caminho de $\neg x$ a $x$ no grafo
        \end{enumerate}
        
        Suponhamos que existam os caminhos $x$ a $\neg x$ e $\neg x$ a $x$ para algum $x \in G$, mas também existe uma associação satisfatível $\rho(x_1,x_2...x_n)$ para a cláusula $\Upsilon$.
        
        Para o caso (i), consideremos que $\rho(x_1,x_2...x_n)$ tal que $x$ é verdadeiro. Assim o caminho de $x$ a $\neg x$ é $x \rightarrow ... \rightarrow \phi \rightarrow \psi \rightarrow \neg x$. Agora, se existe uma aresta entre $X$ e $Y$ do grafo G, então existe $(\neg A \lor B)$ em $\Upsilon$. A aresta de $X$ a $Y$ indica que $X$ é verdadeiro, então $Y$ também deve ser. Como $x$ é verdadeiro, todos os literais entre ele e $\phi$ também devem ser. Da mesma forma, de $\psi$ a $\neg x$ devem ser falsos, pois $\neg x$ é falso. Isso resulta em uma aresta entre $\phi \land \psi$, onde $\phi$ é verdadeiro e $\psi$ é falso. Por consequência, a cláusula $(\neg \phi \lor \psi)$ é falsa, contradizendo a afirmação de $\rho(x_1,x_2...x_n)$ para $\Upsilon$.
        
        Já para o caso (ii), consideremos que $\rho(x_1,x_2...x_n)$ tal que $x$ é falso. Assim, a prova é semelhante ao caso (i).
    \end{enumerate}
    
    Portanto, havendo a existência de um caminho $x$ a $\neg x$ e/ou $\neg x$ a $x$ no grafo, assim podendo utilizar algoritmos como BFS ou DFS, cujos levam tempo polinomial, $\Theta(V + E)$. Assim, provamos que 2-SAT $\in$ P.
  }
  
\item {\bf (2.5 pontos)} Em aula, assumimos que SAT é um problema
  NP-completo (Teorema de Cook-Levin), e a partir deste fato mostramos
  que 3-SAT e CLIQUE também são problemas NP-completos. As reduções
  foram feitas de acordo com o seguinte diagrama:

  $$\xymatrix{
    SAT \ar[d] \\
    3\mbox{-}SAT \ar[d] \\
    CLIQUE 
  }$$
  
  Um ciclo Hamiltoniano é um ciclo simple que visita cada vértice de
  um grafo exatamente uma vez. Considere o problema de decisão
  HAM-CYCLE que pergunta se um dado grafo (não-dirigido) $G$ possui um
  ciclo Hamiltoniano. Mostre que HAM-CYCLE é um problema
  NP-completo. Sua solução deve ser construída a partir de SAT, 3-SAT
  ou CLIQUE. Caso, você não veja como reduzir diretamente HAM-CYCLE a
  partir destes, mas sabe como fazê-lo a partir de um certo problema
  $Q$ então inicialmente mostre que $Q$ é NP-completo a partir de SAT,
  3-SAT ou CLIQUE, e assim por diante. Digamos que você não saiba como
  mostrar que $Q$ é NP-completo diretamente a partir de SAT, 3-SAT ou
  CLIQUE, mas você sabe como fazê-lo a partir de outro problema $Q'$,
  e também sabe como mostrar que $Q'$ é NP-completo a partir de 3-SAT,
  por exemplo. Então o diagrama correspondente à sua solução seria:

$$\xymatrix{
  SAT \ar[d] & & & \\
  3\mbox{-}SAT \ar[d]\ar[r] & Q' \ar[r] & Q \ar[r] & \mbox{HAM-CYCLE}  \\
  CLIQUE & & & 
}$$

E todas as reduções (de 3-SAT para $Q'$, de $Q'$ para $Q$ e de $Q$ para HAM-CYCLE) devem ser detalhadas na sua solução.

\resposta{
    Para provarmos que HAM-CYCLE é NP-Completo, precisamos mostrar que 3-SAT $\leq_p$ HAM-CYCLE.
    
    Assim, consideremos uma instância \textit{I} do 3-SAT, com as variáveis $x_1, x_2, ..., x_n$ e as cláusulas $C_1, C_2, ..., C_n$. Criamos um grafo $G_v$ com que representa as variáveis, enquanto criamos um outro grafo $G_c$ que representa as cláusulas. Vale observar que cada cláusula $C_k$ apresenta-se no formato:
    
    \begin{center}
        $(x_k \lor x_{k+1} \lor x_{k+2})$
    \end{center}
    
    Com isso, unimos as variáveis do grafo $G_v$ com as cláusulas do grafo $G_c$, de forma a criar uma relação do tipo:
    
    \begin{center}
        $\phi = C_1 \land C_2 \land ... \land C_n$.
    \end{center}
    
    Um caminho Hamiltoniano atribui uma associação de verdade para cada variável, dependendo de qual direção a corrente de conexões é transversada.
    
    Para haver um ciclo Hamiltoniano é preciso que cada nó cláusula do grafo seja visitado. Dessa forma, podemos visitar apenas as cláusulas que satisfazem a condição (ao atribuir o valor de verdade para cada uma). Assim se temos um ciclo Hamiltoniano, conseguindo satisfazer a condição da instância \textit{I} do 3-SAT precisa ter um ciclo Hamiltoniano.
  }
  
\item {\bf (2.5 pontos)} Considere o seguinte jogo em um grafo
  (não-dirigido) $G$, que inicialmente contém 0 ou mais bolas de gude
  em seus vértices: um movimento deste jogo consiste em remover duas
  bolas de gude de um vértice $v\in G$, e adicionar uma bola a algum
  vértice adjacente de $v$. Agora, considere o seguinte problema: Dado
  um grafo $G$, e uma função $p(v)$ que retorna o número de bolas de
  gude no vértice $v$, existe uma sequência de movimentos que remove
  todas as bolas de $G$, exceto uma? Mostre que este problema é
  NP-completo. A mesma observação feita no exercício anterior vale
  aqui: a prova deve ser feita a partir de problemas que provamos
  serem NP-completos, e reduções intermediárias, caso existam, devem
  ser incluídas na solução.

  \resposta{
    Escreva aqui sua solução.
  }
  
\item {\bf (2.5 pontos)} Uma fórmula booleana em {\it forma normal conjuntiva com disjunção exclusiva (FNCX)} é uma conjunção de diversas cláusulas, e cada cláusula é uma disjunção exclusiva (XOR) de diversos literais. Lembre-se que a disjunção exclusiva é dada por:

  $$\begin{array}{|l|l|l|}\hline
      a & b & a \oplus b \\ \hline
      V & V & F \\ \hline
      V & F & V \\ \hline
      F & V & V \\ \hline
      F & F & F \\ \hline
  \end{array}$$

  O problema FNCX-SAT pergunta se uma dada fórmula em FNCX é
  satisfatível. Mostre que o problema FNCX-SAT está em $P$, ou então
  que FNCX-SAT é NP-completo. No último caso, a mesma observação feita
  nos dois exercícios anteriores vale aqui: a prova deve ser feita a
  partir de problemas que provamos serem NP-completos, e reduções
  intermediárias, caso existam, devem ser incluídas na solução.

  \resposta{
    Provaremos que FNCX-SAT $\in$ P. Para isso, observaremos o comportamento do XOR ou disjunção exclusiva. Assim, temos a seguinte tabela-verdade:
    
    $$\begin{array}{|l|l|l|1|1|}\hline
      a & b & \neg a \land b & a \land \neg b & (\neg a \land b) \lor (a \land \neg b) \\ \hline
      F & F & F & F & F \\ \hline
      F & V & V & F & V \\ \hline
      V & F & F & V & V \\ \hline
      V & V & F & F & F \\ \hline
  \end{array}$$

    Como pode ser visto, é possível transformar a disjunção exclusiva em uma disjunção de 2 literais. Com isso, podemos tratar o FNCX-SAT como um caso especial do 2-SAT, assim, provando que o FNCX-SAT $\in$ P.
  }
  
\end{enumerate}

\end{document}

%%% Local Variables:
%%% mode: latex
%%% TeX-master: t
%%% End:
